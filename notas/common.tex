\usepackage[utf8]{inputenc}
\usepackage[T1]{fontenc}
\usepackage[brazil]{babel}
\usepackage{geometry}
\usepackage{multicol}
\usepackage{amsthm}
\usepackage{amssymb}
\usepackage{amsfonts}
\usepackage{mathtools}
\usepackage{xcolor}
\usepackage{enumitem}
\usepackage{tikz}
\usepackage{fancyvrb}
\usepackage[framemethod=tikz]{mdframed}
\usepackage{xspace}
\usepackage{xifthen}
\usepackage{listings}
\usepackage{pdfpages}
\usepackage{titlesec}
\usepackage{graphicx}
\usepackage[hidelinks]{hyperref}
\usepackage[portuguese,onelanguage,linesnumbered,noline,noend]{algorithm2e}

% load times font
\usepackage{mathptmx}
\usepackage[scaled=.90]{helvet}
\usepackage{courier}
\DeclareMathAlphabet{\mathcal}{OMS}{cmsy}{m}{n} % uses default mathcal font

% estilo
\sloppy
\geometry{left=2.5cm,right=2cm,top=2cm,bottom=2.5cm}

\DeclareTextFontCommand{\emph}{\em\color{coralerta}}
\newcommand{\alert}[1]{{\color{coralerta}#1}}

% cores
\colorlet{coralerta}{red!60}
\colorlet{cordefs}{blue!70}

% algoritmos
\newcommand{\textonormal}[1]{#1}
\SetFuncArgSty{textonormal}
\SetProgSty{textonormal}
\SetArgSty{textonormal}

\DontPrintSemicolon
\makeatletter
\newcommand{\Algoritmo}[1]{{\color{coralerta}
    \algocf@seteveryparnl{\relax}
    \hspace*{-2ex}#1:\;
}}
\makeatother

\newcommand{\mysty}[1]{\textbf{\color{black!50}\relsize{-2} #1}}
\SetNlSty{mysty}{}{}

% seções
\titleformat*{\section}{\normalfont\sffamily\large\bfseries\color{cordefs}}
\titleformat*{\subsection}{\bfseries\sffamily\color{cordefs}}
\titleformat*{\subsubsection}{\itshape\subsubsectionfont\color{cordefs}}
\titleformat{\paragraph}{\normalfont\color{cordefs}\bfseries}{}{1em}{}

% teoremas
\mdfdefinestyle{mdteostyle}{
  outerlinewidth = 1pt,
  roundcorner =2pt,
  leftmargin = 0,
  rightmargin = 0,
  backgroundcolor = yellow!10,
  outerlinecolor=blue!10!black,
  innertopmargin = \topskip,
  splittopskip = \topskip,
  ntheorem = true,
  nobreak
}
\newtheoremstyle{coloredthm}
  {0pt}% Space above
  {0pt}% Space below
  {\itshape}% Body font
  {}% Indent amount
  {\bfseries\color{cordefs}}% Theorem head font
  {.}% Punctuation after theorem head
  {.5em}% Space after theorem head
  {}% Theorem head spec (can be left empty, meaning ‘normal’)

\newtheoremstyle{coloredthmnonum}
  {0pt}% Space above
  {0pt}% Space below
  {\itshape}% Body font
  {}% Indent amount
  {\bfseries\color{cordefs}}% Theorem head font
  {.}% Punctuation after theorem head
  {.5em}% Space after theorem head
  {\thmname{#1}\thmnote{ (#3)}}% Theorem head spec (can be left empty, meaning ‘normal’)

\newtheoremstyle{coloredprob}
  {0pt}% Space above
  {0pt}% Space below
  {}% Body font
  {}% Indent amount
  {\bfseries\color{cordefs}}% Theorem head font
  {}% Punctuation after theorem head
  {\newline}% Space after theorem head
  {\thmname{#1}\thmnumber{ #2}\thmnote{: {\color{coralerta}#3}}}% Theorem head spec (can be left empty, meaning ‘normal’)

\theoremstyle{coloredthm}
\newmdtheoremenv[style=mdteostyle]{teorema}{Teorema}
\newmdtheoremenv[style=mdteostyle]{corolario}{Corolário}
\newmdtheoremenv[style=mdteostyle]{fato}{Fato}
\newmdtheoremenv[style=mdteostyle]{lema}{Lema}
\newmdtheoremenv[style=mdteostyle]{afirmacao}{Afirmação}
\newmdtheoremenv[style=mdteostyle]{conjectura}{Conjectura}

\theoremstyle{coloredthmnonum}
\newmdtheoremenv[style=mdteostyle]{definicao}{Definição}

\theoremstyle{coloredprob}
\newmdtheoremenv[style=mdteostyle]{auxproblema}{Problema}

% corrige problema com espaçamento quando teorema e prova começa com lista
\newenvironment{problema}[1][]
  {\begin{auxproblema}[#1]\leavevmode\vspace{-\baselineskip}}
  {\end{auxproblema}}

\renewenvironment{proof}[1][Demonstração]{%
  \def\endqedsymbol{\hfill~\qedsymbol}%
  \renewcommand{\qedhere}{\endqedsymbol\gdef\endqedsymbol{\relax}}%
  \noindent\textit{\color{cordefs}#1.}~}{\endqedsymbol}

% nova lista de questões
\newlist{questions}{enumerate}{2}
\setlist[questions,1]{label={\textbf{Questão \arabic*.}},ref=\arabic*,leftmargin=0pt,align=left,itemindent=*}
\setlist[questions,2]{label={\textbf{\textit{(\alph*)}}},itemsep=0pt}
\newlist{cequestions}{enumerate}{1}
\setlist[cequestions]{label={\textbf{\textit{(\alph*)}}\makebox[0pt][l]{\quad C\quad E \quad}},itemsep=0pt,labelsep=1.7cm,leftmargin=!}

% uso: \resposta[texto]{tamanho}
\newcommand{\resposta}[2][Resposta:]{%
  \par\noindent\ifthenelse{\isempty{#1}}{}{{\small #1\\}}%
  \framebox[\linewidth][l]{\rule{0cm}{#2}}\vspace{1pt}}

\newcommand{\videverso}{\vfill\par\hfill{\emph{\footnotesize(vide verso)}}\clearpage}

\newcommand{\Cabecalho}[1]{
  \thispagestyle{empty}
  \begin{center}\small
    Instituto de Computação -- UNICAMP\\
    Algoritmos de Aproximação\\
    \url{http://www.ic.unicamp.br/~lehilton/mo418a/}\\[10pt]
    {\Large\sffamily\color{cordefs} #1}
\end{center}}

\newcommand{\AutoresData}[2]{{\sf\noindent{\color{cordefs}\textbf{Escrito por:}} #1}\hfill{\sf{\color{cordefs}\textbf{Data:}} #2}}

% \CabecalhoLista{nome da lista}{url random}
\newcommand{\ItensInstrucoes}[1][]{

  \raggedright

  \item Os exercícios devem ser submetidos em formato PDF dentro do prazo
    indicado na página \url{https://www.ic.unicamp.br/~lehilton/mo418a/submit/}.\\
    Utilizem a mesma senha passada em sala para as notas de aula.

  \item Os exercícios devem \emph{escritos à mão} e escaneados. Se for
    fotografar, utilize um aplicativo que corrija a perspectiva. Certifique-se
    de que o documento está legível. Faça um rascunho e passe a limpo.

  \ifx\relax#1\relax\else

  \item Só serão aceitas listas com todas questões respondidas, mas serão
    corrigidos \textbf{apenas} os itens sorteados em \url{#1}.

  \fi
}

% listas
\setlist[itemize,1]{label=$\rightarrow$}
\setlist[itemize,2]{label=--}
\setlist[description]{leftmargin=!,labelwidth=1.8cm}

% configura listing e corrige leitura de centos
\lstset{
  showstringspaces=false,
  basicstyle=\ttfamily,
  inputencoding=utf8,
  extendedchars=true,
  literate={á}{{\'a}}1
    {ã}{{\~a}}1
    {â}{{\^a}}1
    {é}{{\'e}}1
    {ê}{{\^e}}1
    {í}{{\'i}}1
    {ó}{{\'o}}1
    {õ}{{\~o}}1
    {ô}{{\^o}}1
    {ú}{{\'u}}1
    {ç}{{\c c}}1
    {vé}{{v\'e}}2,
}

% % USE:
% %      Teclas     Caractere
% %      AlgGr-Z    «
% %      AltGr-X    »
% %      AltGr-4    £
% \begin{Verbatim}[frame=single,commandchars=£«»]
% // exemplo de código
% int misterio(int a, int b) {
%     if (a < b)
%         return 0;
%     else
%         £textbf«ASD» + misterio(a - b, b);
% }
% \end{Verbatim}

% macros de algoritmos
\newcommand{\ptas}{\ensuremath{\mathrm{PTAS}}\xspace}
\newcommand{\pp}{\ensuremath{\mathrm{P}}\xspace}
\newcommand{\np}{\ensuremath{\mathrm{NP}}\xspace}
\newcommand{\maxsnp}{\ensuremath{\mathrm{MAXSNP}}\xspace}

\newcommand{\val}{\text{val}\xspace}
\newcommand{\opt}{\text{opt}\xspace}
\newcommand{\agm}{\text{agm}\xspace}
